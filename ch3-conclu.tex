%---------------------------------------------------------------------------------
\chapter{Conclusions and Future Work }
%---------------------------------------------------------------------------------

\section{Conclusions}
In this report, we first reviewed the mismatch problem of the
statistical speech recognition due to acoustic environment change
and current techniques addressing it. Then the missing feature
technique is discussed in detail. To effectively utilize the
inter-frame information of the speech spectrogram, we proposed to
use the vector autoregressive model for the modeling of speech
spectral vectors in the log Mel filterbank domain. Further more, a
feature compensation technique is proposed based on this model
together with the missing feature theory. The simulation results
using oracle data mask showed the effectiveness of the proposed
feature compensation technique. Specifically, we compare Raj's
MFT-based cluster-based feature compensation technique \cite{Raj04}
with our approach . The results show that the two methods are
comparable if clean training is used, and much better with our
approach when noisy training scheme is used with preprocessing.

In brief, our proposed framework has two novelties. First, we use
the vector autoregressive model in the modeling of speech feature
vectors. Although VAR is used in \cite{Kenny90VAR} for the
derivation of the state-dependent probability of the HMM, it has not
been used in the feature compensation area to our best knowledge.
Second novelty is the use of the noisy training scheme with
preprocessing to minimize the mismatches between the training and
testing environment.

The potential of the proposed VAR based approach is not fully
exploited yet. We believe that further research in this filed will
yield fruitful result. Several directions are discussed in the next
section.
