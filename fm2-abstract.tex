\chapter* {Abstract}
\addcontentsline{toc}{section}{\numberline{}\hspace{-0.35in}{\bf
Abstract}}  % Add the Abstract to the table of contents using the specified format

%Objective
The objective of this research is to develop feature compensation
techniques to make automatic speech recognition (ASR) systems more
robust to noise distortions.
%Importance
The research is important as the performance of ASR systems degrades
dramatically in adverse environments, and hence greatly limits the
speech recognition application deployment.
%Aim
In this report, we aim to build a generic framework for feature
compensation to improve speech recognition accuracy by making speech
features less affected by noises.

%Introduce my approach
The degradation of ASR systems under noisy conditions is due to the
mismatch between the clean-trained acoustical models and noisy
testing speech features presented to the speech recognition engine.
Currently, two general approaches are proposed to reduce this
mismatch. The first is to adapt the acoustical model to the noisy
testing feature, the other is to compensate the noisy testing
feature prior to the recognition. We review existing techniques for
noise robust speech recognition and find that these techniques
generally ignore inter-frame information of the speech signal. We
however believe that inter-frame statistics can contribute to noisy
speech features compensation and hence propose a vector
autoregressive (VAR) model to model speech feature vectors for
speech feature reconstruction by either past or future frames
prediction. We propose two feature compensation schemes based on the
VAR model and the missing feature theory (MFT). Experiments are
carried out using the ground-truth data mask on the AURORA-2
database, and our results show significant improvement to
recognition accuracy. Specifically, our experimental results showed
a relative error rate reductions of 86.51\% and 93.9\% with respect
to the baseline for the subway noise case of test set A and
restaurant noise case of test set B at signal to noise ratio equals
to -5dB.

%Future direction
The proposed VAR modeling framework is a promising research
direction and we will conduct further research to exploit the full
potential of this technique.
